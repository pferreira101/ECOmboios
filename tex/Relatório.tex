% !TEX TS-program = pdflatex
% !TEX encoding = UTF-8 Unicode

% This is a simple template for a LaTeX document using the "article" class.
% See "book", "report", "letter" for other types of document.

\documentclass[11pt]{article} % use larger type; default would be 10pt

\usepackage[utf8]{inputenc} % set input encoding (not needed with XeLaTeX)
\usepackage[portuges]{babel}



%%% PAGE DIMENSIONS
\usepackage{a4}

\usepackage{graphicx} % support the \includegraphics command and options

% \usepackage[parfill]{parskip} % Activate to begin paragraphs with an empty line rather than an indent

%%% PACKAGES
\usepackage{xcolor} % cores
\usepackage{booktabs} % for much better looking tables
\usepackage{array} % for better arrays (eg matrices) in maths
\usepackage{paralist} % very flexible & customisable lists (eg. enumerate/itemize, etc.)
\usepackage{verbatim} % adds environment for commenting out blocks of text & for better verbatim
\usepackage{subfig} % make it possible to include more than one captioned figure/table in a single float
% These packages are all incorporated in the memoir class to one degree or another...

%%% HEADERS & FOOTERS
\usepackage{fancyhdr} % This should be set AFTER setting up the page geometry
\pagestyle{fancy} % options: empty , plain , fancy
\renewcommand{\headrulewidth}{0pt} % customise the layout...
\lhead{}\chead{}\rhead{}
\lfoot{}\cfoot{\thepage}\rfoot{}

%%% SECTION TITLE APPEARANCE
\usepackage{sectsty}
\allsectionsfont{\sffamily\mdseries\upshape} % (See the fntguide.pdf for font help)
% (This matches ConTeXt defaults)

%%% ToC (table of contents) APPEARANCE
\usepackage[nottoc,notlof,notlot]{tocbibind} % Put the bibliography in the ToC
\usepackage[titles,subfigure]{tocloft} % Alter the style of the Table of Contents
\renewcommand{\cftsecfont}{\rmfamily\mdseries\upshape}
\renewcommand{\cftsecpagefont}{\rmfamily\mdseries\upshape} % No bold!

%%% END Article customizations

%%% The "real" document content comes below...

\title{Comboios de Portugal}
\author{The Author}
%\date{} % Activate to display a given date or no date (if empty),
         % otherwise the current date is printed 

\begin{document}
\maketitle


\newpage
\tableofcontents
\newpage

\section{Definição do Sistema}


\subsection{Contexto de aplicação do sistema}
Com a necessidade de reduzir a poluição causada pelos veículos pessoais, os transportes públicos são uma das soluções para tal, mais concretamente os Comboios de Portugal.


\subsection{Fundamentação da implementação da base de dados}

A necessidade de manter informação de todos os clientes, todos os comboios, todos os funcionários, assim como de todas as linhas e viagens faz com que seja necessário criar um sistema de bases de dados para suportar tudo isto.


\subsection{Análise da viabilidade do processo}



\section{Levantamento e Análise de Requisitos}

\subsection{Método de levantamento e de análise de requisitos adotado}

\subsection{Requisitos levantados}

\subsubsection{Requisitos de descrição}

\subsubsection{Requisitos de exploração}

\subsubsection{Requisitos de controlo}

\subsection{Análise geral dos requisitos}



\section{Modelação Conceptual}

\subsection{Apresentação da abordagem de modelação realizada}

\subsection{Identificação e caracterização das entidades}

\subsection{Identificação e caracterização dos relacionamentos}

\subsection{Identificação e caracterização das Associação dos Atributos com as Entidades e Relacionamentos}

\subsection{Detalhe ou generalização de entidades}

\subsection{Apresentação e explicação do diagrama ER}

\subsection{Validação do modelo de dados com o utilizador}




\section{Modelação Lógica}

\subsection{Construção e validação do modelo de dados lógico}

\subsection{Desenho do modelo lógico}

\subsection{Validação do modelo através da normalização}

\subsection{Validação do modelo com interrogações do utilizador}

\subsection{Validação do modelo com as transações estabelecidas}

\subsection{Reavaliação do modelo lógico (se necessário)}

\subsection{Revisão do modelo lógico com o utilizador}




\section{Implementação Física}

\subsection{Seleção do sistema de gestão de bases de dados}

\subsection{Tradução do esquema lógico para o sistema de gestão de bases de dados escolhido em SQL}

\subsection{Tradução das interrogações do utilizador para SQL (alguns exemplos)}

\subsection{Tradução das transações estabelecidas para SQL (alguns exemplos)}

\subsection{Escolha, definição e caracterização de índices em SQL (alguns exemplos)}

\subsection{Estimativa do espaço em disco da base de dados e taxa de crescimento anual}

\subsection{Definição e caracterização das vistas de utilização em SQL (alguns exemplos)}

\subsection{Definição e caracterização dos mecanismos de segurança em SQL (alguns exemplos)}

\subsection{Revisão do sistema implementado com o utilizador}



\section{Cunclusões e Trabalho Futuro}



\section{Referências Bibliográficas (em formato Harvard)}


\textcolor{red}{ANEXOS}







\end{document}
